\begin{frame}[fragile]{}
\begin{tikzpicture}[remember picture,overlay]
%\draw[draw=rwthlightgray] (0, 0) grid (110.08cm, -60cm);
\end{tikzpicture}
%
\vspace*{-0.5cm}%
\begin{columns}[onlytextwidth]%
%
\begin{column}{26.02cm}%
\begin{tikzpicture}[remember picture, overlay]
\foreach \i in {0,3} {
\fill[rwthlightgray] ({28.02cm*\i}, -.1ex) rectangle +(26.02cm, -62cm);
%\draw[rwth100blue] ({28.02cm*\i}, -.1ex) rectangle +(26.02cm, -62cm);
}
\foreach \i in {1} {
\fill[rwthlightgray] ({28.02cm*\i}, -25cm+.3ex) rectangle +(26.02cm, -37cm-.4ex);
%\draw[rwth100blue] ({28.02cm*\i}, -22cm+.3ex) rectangle +(26.02cm, -40cm-.4ex);
}
\foreach \i in {2} {
\fill[rwthlightgray] ({28.02cm*\i}, -35cm+.7ex) rectangle +(26.02cm, -27cm-.8ex);
%\draw[rwth100blue] ({28.02cm*\i}, -35cm+.7ex) rectangle +(26.02cm, -27cm-.8ex);
}
\end{tikzpicture}%
%
\begin{block}{Molecular Dynamics (MD)}%
Molecular dynamics potentials specify how atoms attract and repulse each other and account for the majority of the simulation runtime.
They are given as functions of all atom positions:
\hfill $V(\mathbf{x}_1, \dots, \mathbf{x}_N)$.\\
And forces can be derived using
\hfill $\mathbf{F}_i = -\nabla_{\mathbf{x}_i} V$.\\
Simplest and most popular potential (pairwise) depends on distance:
$$V=\sum_i\sum_{j, r_{ij} < r_c} f(r_{ij})$$

Many-body potentials are required to model some materials, and depend on more than just distance:

\begin{center}%
\begin{tikzpicture}
%\draw[rwthlightgray] (0, 0) grid (26, 10);
%\clip (0, 0) rectangle (26, 10);
\node[circle, minimum size=2.0cm] (k) at (3, 7) {k};
\node[circle, minimum size=2.0cm] (i) at (9, 3) {i};
\node[circle, minimum size=2.0cm] (j) at (17, 7) {j};
\node[circle, minimum size=2.0cm] (l) at (23, 3) {l};
\draw[draw=rwthgreen, line width=5pt] (j) -- (l);
\draw[draw=rwthturquoise, line width=5pt] (k) -- (i);
\pic ["$\theta_{ijl}$", line width=5pt, draw=rwthmagenta, ->, angle radius=2cm, angle eccentricity=1.5] {angle =  i--j--l};
\pic ["$\theta_{jik}$", line width=5pt, draw=rwthmaygreen, ->, angle radius=2cm, angle eccentricity=1.5] {angle =  j--i--k};
\draw[draw=rwthpetrol, line width=5pt] (i) -- (j);
\draw[draw=rwthorange, line width=5pt, ->] ($ (i)!0.5!(j) $) circle (1.2cm);
\draw[draw=rwthorange, line width=5pt, ->] ($ (i)!0.5!(j) + (90:1.2cm) $) -- ($ (i)!0.5!(j) + (95:1.2cm) $);
\node at ($ (i)!0.5!(j) + (0, 2cm) $) {$\omega_{kijl}$} ;
\draw[draw=rwthpetrol, line width=5pt] ($(i)!0.5!(j)$) -- (j);
%\draw[draw=rwthmaygreen, line width=5pt] ($ (k) + (0, 10pt) $) -- ($ (i) + (0, 10pt) $) -- ($ (j) + (0, 10pt) $);
%\draw[draw=rwthorange, line width=5pt] ($ (i) + (0, -10pt) $) -- ($ (j) + (0, -10pt) $) -- ($ (l) + (0, -10pt) $);


\node[circle, line width=3pt, draw=rwthred, fill=white, minimum size=2.0cm] (k) at (3, 7) {k};
\node[circle, line width=3pt, draw=rwthbordeauxred, fill=white, minimum size=2.0cm] (i) at (9, 3) {i};
\node[circle, line width=3pt, draw=rwthviolet, fill=white, minimum size=2.0cm] (j) at (17, 7) {j};
\node[circle, line width=3pt, draw=rwthpurple, fill=white, minimum size=2.0cm] (l) at (23, 3) {l};
\end{tikzpicture}%
\end{center}%
The compiler is particularly intended for these potentials, since implementing them goes beyond taking the derivative of a scalar function.
\end{block}%
%
%
\begin{block}{Selected Potentials from LAMMPS}
\begin{center}
\begin{tabular}{llll}
\bfseries Name &\bfseries  LOC-R &\bfseries  LOC-O &\bfseries  Structure\\
LJ & 640 & +480 & {\small $\sum_i\sum_j f(i, j)$}\\
Stillinger-Weber & 600 & +1250 & {\small $\sum_i\sum_j\sum_k f(i,j,k)$}\\
EAM & 840 & +820 & {\small $\sum_i f(i, \sum_j g(i, j))$}\\
Tersoff & 800 & +1450 & {\small $\sum_i\sum_j f(i, j, \sum_k g(i, j, k)$}\\
MEAM & 890 & \ding{55}  & {\small $\sum_i f(i, \sum_j g(i, j))$}\\
ADP & 940  & \ding{55}  & too complex to show\\
BOP & 5950 & \ding{55}  & too complex to show\\
(AI)REBO & 4240 & +4550 & too complex to show\\
COMB3 & 3560   & \ding{55} & too complex to show\\
ReaxFF & 10880 & \ding{55} & too complex to show
\end{tabular}
\end{center}
LOC-R: Lines of code of the regular code.\\
LOC-O: Extra LOC in the optimized/vectorized code.
\end{block}%
\begin{block}{Motivation}%
%
Many-body potentials are complex to implement, and require large effort to optimize.
Less popular potentials will likely never be optimized manually.


And even if there is an implementation, it can be hard to assure that it is correct.


To tackle these issues, this work introduces a Domain Specific Language for molecular dynamics potentials and a corresponding compiler to derive high performance implementations.

\end{block}%
%
\end{column}%
%
\begin{column}{2cm}%
\end{column}%
%
\begin{column}{26.02cm}%
%
%
%\begin{center}%
%\vspace{-1.5cm}%
%\begin{tikzpicture}
%\draw[white] (0, 0) grid (54.04cm, 30cm-.7ex);
%\end{tikzpicture}
%\end{center}%
%

\begin{block}{Example: The Tersoff Potential}

\vspace*{-0.5cm}

\begin{align*}
V &= \sum_i\sum_j f_C(r_{ij}) [ f_R(r_{ij}) + b_{ij} f_A(r_{ij}) ]\\
f_C(r) &= \begin{cases}
  0, & r < R - D\\
  \frac{1}{2} - \frac{1}{2}\sin(\frac{\pi}{2} (r - R) / D), & R - D < r < R + D\\
  1 & r > R + D
\end{cases}\\
f_R(r) &= A\exp(-\lambda_1 r)\\
f_A(r) &= -B\exp(-\lambda_2 r)\\
b_{ij} &= (1 + \beta^n\zeta_{ij}^n)^{1/(2n)}\\
\zeta_{ij} &= \sum_k f_C(r_{ik}) g(\theta_{jik}) \exp(\lambda_3^m (r_{ij} - r_{ik})^m)\\
g(\theta) &= \gamma (1 + c^2/d^2 - c^2/(d^2 + (\cos(\theta) - \cos(\theta_0))^2))\\
\end{align*}


Parameters: $R$, $D$, $A$, $\lambda_1$, $B$, $\lambda_2$, $\beta$, $n$, $\lambda_3$, $m$, $\gamma$, $c$, $d$, $\theta_0$.

\vspace*{1cm}

\end{block}

\begin{block}{Challenges}%
\begin{minipage}{24.02cm}
%\vspace*{.5em}
\begin{itemize}
\item Picking a vectorization strategy suitable to the neighbor list lengths.
\item Reducing the amount of recalculation for the force computation.
\item Introducing reduced accuracy modes.
\item Applying a number of vectorization idioms.
\item Performing optimizations that normal compilers can not apply as they lack MD-specific knowledge.
\end{itemize}
\end{minipage}
\end{block}%
\begin{block}{Approach}%
%
\vspace{-0.5cm}
\begin{center}
\begin{tikzpicture}
%\draw (-13, 18) rectangle (13, 0);
\node (a) at (0, 19) {};
\node[text width=20cm, align=center, fill=rwthpetrol,   rectangle, text=white] (b) at (0, 16) {1. Parse Input DSL\rule[-0.2\baselineskip]{0pt}{1\baselineskip}};
\node[text width=20cm, align=center, fill=rwthmaygreen, rectangle] (c) at (0, 12) {2. Force Derivation\rule[-0.2\baselineskip]{0pt}{1\baselineskip}};
\node[text width=20cm, align=center, fill=rwthmagenta,  rectangle, text=white] (d) at (0, 8) {3. Optimization\rule[-0.2\baselineskip]{0pt}{1\baselineskip}};
\node[text width=20cm, align=center, fill=rwthorange,   rectangle] (e) at (0, 4) {4. Code Generation\rule[-0.2\baselineskip]{0pt}{1\baselineskip}};
\node (f) at (0, 0) {};
\draw[-{Latex[length=5mm]}] (a) -- (b) node[right, pos=0.5] {Input text};
\draw[-{Latex[length=5mm]}] (b) -- (c) node[right, pos=0.5] {Functional repr.};
\draw[-{Latex[length=5mm]}] (c) -- (d) node[right, pos=0.5] {Imperative repr.};
\draw[-{Latex[length=5mm]}] (d) -- (e) node[right, pos=0.5] {Imperative repr.};
\draw[-{Latex[length=5mm]}] (e) -- (f) node[right, pos=0.5, text width=10cm] {Output text, including ``bookkeeping''};
\end{tikzpicture}
\end{center}
\end{block}%
%
\end{column}%
%
\begin{column}{2cm}%
\end{column}%
%
\begin{column}{26.02cm}%
%
%\vspace*{30cm}%
\setbeamercolor{block title}{bg=rwthpetrol,fg=white}
\begin{block}{DSL for Tersoff Potential}%
%
\vspace{-0.5cm}%
\begin{lstlisting}[language=potc, basicstyle=\small\ttfamily]
energy 1 / 2 * sum(i : all_atoms)
 sum(j : neighbors(i, R(i, j, j) + D(i, j, j)))
  V(i, j);
function V(i : atom; j : atom) = 
 f_C(i, j, r(i, j)) * 
 (f_R(i, j, r(i, j)) + 
  b(i, j) * f_A(i, j, r(i, j)));
function f_C(i : atom_type; j : atom_type;
             k : atom_type; r : distance) = 
 implicit(i : i; j : j; k : k)
  piecewise(
   r <= R - D : 1; 
   R - D < r < R + D : 
    1 / 2 - 1 / 2 * sin(pi / 2 * (r - R) / D); 
   r >= R + D : 0);
# ...
function b(i : atom; j : atom) =
 (1 + beta(i, j) ^ n(i, j) * 
  zeta(i, j) ^ n(i, j)) ^ (-1 / (2 * n(i, j)));
function zeta(i : atom; j : atom) =
 sum(k : neighbors(i, R(i, j, k) + D(i, j, k), j))
  implicit(i : i; j : j; k : k)
   f_C(r(i, k)) * g(theta(j, i, k)) * 
   exp(lambda_3 ^ m * (r(i, j) - r(i, k)) ^ m);

parameter A(i : atom_type; j : atom_type) = file(1);
# ...
parameter gamma(i : atom_type; j : atom_type;
                k : atom_type) = file(5);
# Total: ~ 30 LOC, while LAMMPS impl > 500 LOC
\end{lstlisting}

\end{block}
%
\setbeamercolor{block title}{bg=rwthmaygreen,fg=black}
\begin{block}{Force Derivation}%
There is one variable (the potential energy) that we want to take many derivatives of.

An adjoint-based approach makes this calculation very straightforward, essentially ``interpreting`` the functional representation and emitting the corresponding code.
\end{block}%
%
\setbeamercolor{block title}{bg=rwthmagenta,fg=white}
\begin{block}{Optimization}%
%
\vspace*{-0.5cm}%
\begin{itemize}
\item Inlining
\item Loop/Conditional Fusion
\item Dead Code Elimination
\item Common Subexpression Elimination
\item Arithmetic Improvements (e.g Constant Folding)
\item Loop Invariant Code Motion
\item Vectorization (partial)
\color{magenta}
\item Check for 0
\item Caching
\item Batching
\item Accuracy
\end{itemize}

\end{block}
\end{column}%
%
\begin{column}{2cm}%
\end{column}%
%
\begin{column}{26.02cm}%
%
\setbeamercolor{block title}{bg=rwthorange,fg=black}
\begin{block}{Code Generation for LAMMPS}%
For now, the compiler generates source code that can be compiled into LAMMPS.
LAMMPS is a popular, extensible, open-source, highly scalable molecular dynamics code developed by Sandia National Labs.
It contains implementations for a number of many-body potentials, in various implementation variants (threaded, vectorized, GPU, \dots).
In addition LAMMPS provides benchmarks for all its potentials, which can be used to evaluate our results.
\end{block}%
\setbeamercolor{block title}{bg=rwthdarkblue,fg=white}
\begin{block}{Performance}%
\begin{center}
\pgfplotsset{compat=1.11,
    /pgfplots/ybar legend/.style={
    /pgfplots/legend image code/.code={%
       \fill[##1,/tikz/.cd,yshift=-0.25em]
        (0cm,0cm) rectangle (0.4em,0.8em);},
   },
}
\begin{tikzpicture}
\draw[rwthlightgray] (0, 0) grid (25, 10);
\node (t) at (6.5, 5.0) {};
\node (s) at (18.5, 5.0) {};
\begin{axis}[legend cell align=left,
    legend pos=north east,
    legend style={draw=none},
    ylabel={Time (s)},
    label style={font=\footnotesize},
    tick label style={font=\tiny},
    x tick label style={font=\small},
    y label style={at={(axis description cs:-0.08,.5)},anchor=south},
    axis background/.style={fill=white},
    legend style={fill=none},
    ybar, bar width=30pt,
    enlarge x limits=1.2,
    ymin=0,
    symbolic x coords={Hand, Gen},
    xtick={Hand, Gen},
    title={Tersoff\rule[-0.2\baselineskip]{0pt}{1\baselineskip}},
    at={(t)}, anchor=center,
    height=8.5cm, width=11cm
]
\addplot[draw=none, fill=rwthdarkblue] coordinates { (Hand, 30.2) (Gen, 24.8) };
\addplot[draw=none, fill=rwthmagenta]  coordinates { (Hand, 9.5) (Gen, 5.5) };
\legend{ Reg, Vec }
\end{axis}

\begin{axis}[legend cell align=left,
    legend pos=north east,
    legend style={draw=none},
    ylabel={Time (s)},
    label style={font=\footnotesize},
    tick label style={font=\tiny},
    x tick label style={font=\small},
    y label style={at={(axis description cs:-0.08,.5)},anchor=south},
    axis background/.style={fill=white},
    legend style={fill=none},
    ybar, bar width=30pt,
    enlarge x limits=1.2,
    ymin=0,
    symbolic x coords={Hand, Gen},
    xtick={Hand, Gen},
    title={Stillinger-Weber},
    at={(s)}, anchor=center,
    height=8.5cm, width=11cm
]
\addplot[draw=none, fill=rwthdarkblue] coordinates { (Hand, 20.9) (Gen, 18.9) };
\addplot[draw=none, fill=rwthmagenta]  coordinates { (Hand, 3.1) (Gen, 9.3) };
\legend{ Reg, Vec }
\end{axis}
\pgfresetboundingbox
\path [use as bounding box] (0, 0) rectangle (25, 10);
\end{tikzpicture}
\small KNL, double precision, single core.
\vspace*{-0.5em}
\end{center}
\end{block}%
\begin{block}{Future Work}%
\begin{minipage}{24.02cm}
\begin{itemize}
\item Extend the range of supported potentials by adding the features they require.
\item Target additional parts of LAMMPS, for example to enable parallelization via KOKKOS.
\item Target the KIM API to make the generated output more generally interoperable.
\item Perform other symbolic manipulations, e.g. calcu-\\late parameter derivatives, or accurate Hessians.
\end{itemize}
\end{minipage}
\end{block}%
\begin{block}{Conclusion}
PotC can be a valuable tool for molecular dynamics software authors and users.
It allows them to \dots

\begin{minipage}{24.02cm}
\vspace*{.5em}
\begin{itemize}
\item Quickly prototype novel potentials.
\item Test other implementations against mechanically derived ``ground truth''.
\item Generate mechanically optimized implementations for the ``tail'' of potentials that lacks the popularity to be ever optimized manually.
\item Have an unambiguous and concise way to transfer and write about potentials.
\item Avoid redundancies due to differentiation.
\item Avoid optimizing code by hand.
\end{itemize}
\end{minipage}
\end{block}
%\begin{block}{References}%
%many-body paper plimpton/thompson, implementing some many-body potentials papers
%\end{block}%
%
\end{column}%
%
\end{columns}%
%
\end{frame}


